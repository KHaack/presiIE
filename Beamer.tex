%This Source Code Form is subject to the terms of the Mozilla Public License, v. 2.0. If a copy of the MPL was not distributed with this file, You can obtain one at http://mozilla.org/MPL/2.0/.
%%%
% Minimal working example for the beamer theme "minimal"
% Last Update: 15-02-2017
% In case of questions/problems/proposals you can write, call or visit me
% http://cs.uni-paderborn.de/cuk/personal/sascha-brauer/
%%%
\documentclass{beamer}

%% Use the handout option to color-invert decorations
\usetheme{minimal}

\usepackage{color}
\usepackage{lmodern}
\usepackage{outlines}

\usepackage{graphicx}
\usepackage{media9}

\title{Thema 7\\Hypertext-Systeme 1}

\author{Kevin Haack}
\institute{Universität Paderborn}

\date{\today}

\setbeamertemplate{section in toc}[ball unnumbered]


\begin{document}

\begin{frame}
  \titlepage
\end{frame}

\begin{frame}{Übersicht}
\tableofcontents
\end{frame}

\section{Abgrenzung}
\begin{frame}{Abgrenzung}
	\begin{itemize}
		\item Memex (1945)
		\item Hypertext-Systeme 2
		\item closed hypertext systems
	\end{itemize}
\end{frame}

\begin{frame}{Hyperspace ohne Internet}
	\begin{itemize}
		\item Hypertext kann man sich kaum vorstellen ohne Internet
		\item wie könnte das sonst aussehen?
	\end{itemize}
\end{frame}


\section{Geschichte}
\begin{frame}{Geschichte}
\begin{itemize}
	\item 1945: \textbf{MEMEX / As we may think}
	\item 1960: \textbf{Xanadu}
	\item 1967: \textbf{HES}
	\item 1968: The mother of all demos, \textbf{FRESS}, \textbf{NLS}
	\item 1980: ENQUIRE am CERN
	\item 1982: Guide
	\item 1983: \textbf{HyperTIES}
	\item 1985: \textbf{Document Examiner}
	\item 1987: HyperCard für MAC
	\item 1988: Hypertext on Hypertext
	\item 1989: Information Management: a Proposal
	\item 1990: WinHelp, Storyspace, Hypertext hands-on!
\end{itemize}
\end{frame}












\section{Funktionen}
\begin{frame}{Funktionen}
\begin{itemize}
	\item einzelne Systeme betrachten
	\item warum durchgesetzt, warum nicht?
	\item Bedienmöglichkeiten
	\item Vergleich zu Memex, welche Konzepte werden abgedeckt?
	\item Was haben die Systeme versprochen und was gehalten?
	\item Vannevar Bush points out, that classical filing methods like sorting by alphabetical order are artificial and do not correspond to the way humans think. [Bush 1945?]
	\item more natural
\end{itemize}
\end{frame}

\begin{frame}{Funktionen - Beispiele}
\begin{itemize}
	\item tote Links
	\item lost in Hyperspace
	\item Mensch-Computer-Interaktion
	\item Wechselwirkung: Bedarf - Entwicklung
\end{itemize}
\end{frame}



\begin{frame}
\frametitle{Memex}
\begin{itemize}
	\item maschinellen Unterstützung des menschlichen Gedächtnisses und des assoziativen Denkens
	\item Schreibtisch
	\item Kombination von elektromechanischen Kontrollen und Mikrofilmgeräten
	\item Illustrationen im Life Magazine 19. November 1945
	\item kopfmontierte Kamera sowie eine Schreibmaschine, die über Spracherkennung verfügen und die Texte mittels Sprachsynthese vorlesen soll. 
\end{itemize}
\end{frame}

\begin{frame}
	\frametitle{Memex}
	\framesubtitle{Funktionen}
	\begin{itemize}

		\item Seiten durch Verknüpfungen (associations) aufeinander verweisen zu lassen.
		\item mit Hebeln vor- und zurückblättern sowie Dokumente speichern und wieder aufrufen
		\item berührungssensitiven Bildschirmen 

	\end{itemize}

	\begin{figure}[htbp]
		\centering
		\includegraphics[width=0.4\textwidth]{images/memex}
	\end{figure}

\end{frame}

\begin{frame}
\frametitle{Xanadu}
\begin{itemize}
	\item Framework
	\item hypertext offers several different branches to assemble the meaning behind the written text
	\item more then "decent writing system" [Nelson 1974?, p. DM 59]
	\item "docuverse" - ein elektronische universale Bibliothek
\end{itemize}

\begin{figure}[htbp]
	\centering
	\includegraphics[width=0.25\textwidth]{images/xanadu}
\end{figure}

\end{frame}

\begin{frame}
\frametitle{Xanadu}
\framesubtitle{Funktionen}
\begin{itemize}

	\item Kommentare, Notizen und Verknüpfungen – between places in documents, and leave them there for others [Nelson 93, p. 10]
	\item royalty and credit to the originator [Nelson 93, p. 10]
	

\end{itemize}
\end{frame}

\begin{frame}
\frametitle{NLS: oN-Line System}
\begin{itemize}
	\item “augment human intellect”
	\item computer screens can and should be used to display text [Engelbart 1962]
	\item directly interact [Engelbart 1962]
	\item first time-sharing computers (six terminals)
\end{itemize}

\begin{figure}[htbp]
	\centering
	\includegraphics[width=0.7\textwidth]{images/nls}
\end{figure}

\end{frame}

\begin{frame}
\frametitle{NLS: oN-Line System}
\framesubtitle{Funktionen}
\begin{itemize}

	\item Links 
	\item mouse, windows, interactive text editing
	

\end{itemize}
\end{frame}

\begin{frame}
\frametitle{HES: Hypertext Editing System}
\begin{itemize}
	\item 1967
	\item Andries van Dam und Ted Nelson, Brown University
	\item IBM/360 Model 50 Mainframe
	\item Dokumentation der Apollo Missionen, NASA
\end{itemize}

\begin{figure}[htbp]
	\centering
	\includegraphics[width=0.40\textwidth]{images/hes}
\end{figure}

\end{frame}

\begin{frame}
\frametitle{HES: Hypertext Editing System}
\framesubtitle{Funktionen}
	\begin{itemize}
		\item Userinterface / Bedienmöglichkeiten
		\begin{itemize}
			\item Lightpenning und Tastatur
			\item Graph Darstellung
		\end{itemize}
		\item Links / Strukturen
		\begin{itemize}
			\item Kontrollinformationen für eine lineare Darstellung
			\item Pointer zu Text Fragmenten
			\item Textpassagen mit Labels referenzierbar
		\end{itemize}
	\end{itemize}
\end{frame}

\begin{frame}
\frametitle{FRESS: File Retrieval and Editing System}
\framesubtitle{Funktionen}
\begin{itemize}
	\item 1968
	\item Andries van Dam, Bob Wallace und Studenten
	\item Kommerzielle Betriebssysteme
	\item Weiterentwicklung von HES
	\item Userinterface / Bedienmöglichkeiten
	\begin{itemize}
		\item Backtrack durch die Links
		\item Windows
		\item UNDO Feature
		\item auf PDS-1 (Minicomputer): Light Pen und Fußpedal
	\end{itemize}
	\item Links / Strukturen
	\begin{itemize}
		\item one-way Links (tag)
		\item bi-direktionale Links (jumps)
		\item unbegrenzte Dokumentgrößen
	\end{itemize}
\end{itemize}
\end{frame}

\begin{frame}
\frametitle{NoteCards}
\begin{itemize}
	\item xxxx
\end{itemize}

\end{frame}
\begin{frame}
\frametitle{Document Examiner}
\begin{itemize}
	\item 1985
	\item Symbolics Inc.
	\item Symbolics Handbücher
	\item Concordia als Editor
\end{itemize}

\begin{figure}[htbp]
	\centering
	\includegraphics[width=0.5\textwidth]{images/documentExaminer}
\end{figure}

\end{frame}

\begin{frame}
\frametitle{Document Examiner}
\framesubtitle{Funktionen}
	\begin{itemize}
		\item Userinterface / Bedienmöglichkeiten
		\begin{itemize}
			\item An die Recherche in Handbüchern angepasst
			\item Content area, hinzu, Bookmarks, Command region
			\item Ein Klick auf einen Link fügt zu Kandidaten hinzu
		\end{itemize}
		\item Links / Strukturen
		\begin{itemize}
			\item Inspiriert von NLS, Xanadu und HES
			\item Records enthalten Titel und Beschreibung
			\item Record hat ID
			\item Sequenzen von Records
		\end{itemize}
	\end{itemize}
\end{frame}
\begin{frame}
\frametitle{HyperTIES}
\begin{itemize}
	\item simplified approach in browsing the hypertext [Nielsen 90, p. 120]
	\item MS-DOS
	\item click on hyperlinks if a mouse or a touch screen is present.
	\item all interactions can be performed with the arrow keys
	\item article. It has a title and a short description about its content. The title is used to automatically place links wherever the very same text phrase appears in other articles
	\item description is used as a preview for a link
\end{itemize}

\end{frame}

\begin{frame}
\frametitle{HyperCard}
\begin{itemize}
	\item rectangular areas on top of the text layer
\end{itemize}

\end{frame}



\begin{frame}
\frametitle{Storyspace}
\begin{itemize}
	\item 1990 by Mark Bernstein
	\item Macintosh (Portierung für Windows)
	\item Hyperlinks do not have to be coded, they are created and directly manipulated with the mouse
	\item diagram mode visually reveals the logical structure of arguments
	\item network of writing spaces
	\item nodes inside a writing space can link to any other writing space
	\item Images can be placed into the flow of text
	\item Links are not highlighted
	\item links can point to several targets at the same time
	\item conditions to links
\end{itemize}

\end{frame}











\end{document}